\documentclass[12pt,a4paper,titlepage]{article}
\usepackage[spanish]{babel}
\usepackage{listings}
\usepackage{graphicx}
\usepackage{titlepic}
\usepackage{lipsum}
\usepackage[svgnames]{xcolor}
\usepackage[tikz]{bclogo}
\usepackage{minted}

% Default fixed font does not support bold face
\DeclareFixedFont{\ttb}{T1}{txtt}{bx}{n}{12} % for bold
\DeclareFixedFont{\ttm}{T1}{txtt}{m}{n}{12}  % for normal

% Custom colors
\usepackage{color}
\definecolor{deepblue}{rgb}{0,0,0.5}
\definecolor{deepred}{rgb}{0.6,0,0}
\definecolor{deepgreen}{rgb}{0,0.5,0}

% Python style for highlighting
\newcommand\pythonstyle{\lstset{
language=Python,
basicstyle=\ttm,
otherkeywords={self},             % Add keywords here
keywordstyle=\ttb\color{deepblue},
emph={MyClass,__init__},          % Custom highlighting
emphstyle=\ttb\color{deepred},    % Custom highlighting style
stringstyle=\color{deepgreen},
frame=tb,                         % Any extra options here
showstringspaces=false            % 
}}

% Python environment
\lstnewenvironment{python}[1][]
{
\pythonstyle
\lstset{#1}
}
{}

% Python for external files
\newcommand\pythonexternal[2][]{{
\pythonstyle
\lstinputlisting[#1]{#2}}}

% Python for inline
\newcommand\pythoninline[1]{{\pythonstyle\lstinline!#1!}}


  
\title{Universidad de Mor\'on\\ Trabajos pr\'acticos de sistemas operativos distribuidos(444)}

\author{Mat\'ias Fredes \\Matricula: 38010252}
 
\begin{document}
    \maketitle
    \newpage
    \begin{bclogo}[logo=\bcattention, couleurBarre=red, noborder=true, 
               couleur=LightSalmon]{Importante!}
    Este trabajo pr\'actico fue desarrollado con el lenguaje de programaci\'on Python.\\
    Para instalar Python en su computadora debe dirigirse a la siguiente URL: https://www.python.org/downloads/.\\
    Para ejecutar los programa desde la consola, debe posicionarse en el directorio donde se encuentra el archivo y ejecutar el siguiente comando:\begin{minted}{bash}
  $ python nombreArchivo.py
    \end{minted}
\end{bclogo}
    \section{Mini servidor de comandos}
    
  
    
    
    \noindent Desarrollar un mini servidor de comandos Linux y Un mini cliente Telnet para el Mini Servidor.\\
    Es muy simple, el Cliente Mini Telnet envía un comando al Servidor, el Servidor lo ejecuta y le devuelve la salida del comando al Cliente, el cliente lo muestra por pantalla.\\
    Desarrollar un servidor secuencial y un servidor concurrente utilizando procesos pesados(fork) y procesos livianos(thread).
    \subsection{Cliente telnet}
    \begin{python}
    import socket
    s = socket.socket()
    s.connect(('192.168.239.132', 9999))
    estado = True
    while estado:
        mensaje = raw_input("> ")
        s.send(mensaje)
        mensajeServidor = s.recv(4096)
        if mensajeServidor == "quit":
    	    s.close()
    	    estado = False
        else:
            print mensajeServidor
    \end{python}
    \subsection{Servidor de comandos secuencial}
    \begin{python}
    import socket
    from subprocess import check_output

    class Server:
        def __init__(self):
        	self.host, self.port = '', 9999
        	self.socket = socket.socket()
        	self.socket.bind((self.host, self.port))
    
        def send(self, msg):
        	if type(msg) == str: self.socketCliente.send(msg)
        	elif type(msg) == list or tuple: self.socketCliente.send('\n'.join(msg))
    
        def recv(self):
        	comando = self.socketCliente.recv(4096).strip()
            if comando == "quit":
                self.exit()
            else: return comando
    
        def exit(self):
           self.send("quit");
           self.socketCliente.close(); self.run()
        	# cierro la conexion y abro una nueva
        
        def ejecutarComando(self, comando):
            return check_output(comando)    
    
        # main runtime
        def run(self):
            self.socket.listen(1)
        	self.socketCliente, self.addr = self.socket.accept()
        	while True:
                comando = self.recv()
                print comando
                retorno = self.ejecutarComando(comando)
                print retorno
                self.send(retorno)
    
    
    S = Server()
    S.run()
    \end{python}
    
     \subsection{Servidor de comandos concurrente con procesos pesados(fork)}
    \begin{python}
import socket
import os
import threading
from subprocess import check_output

class Server(object):
    def __init__(self):
    	self.host, self.port = '', 9999
    	self.socket = socket.socket()
    	self.socket.bind((self.host, self.port))

    def send(self, msg,socketCliente):
    	if type(msg) == str: socketCliente.send(msg)
    	elif type(msg) == list or tuple: socketCliente.send('\n'.join(msg))

    def recv(self,socketCliente):
    	comando = socketCliente.recv(4096).strip()
        if comando == "quit":
            self.exit(socketCliente)
        else: return comando

    def exit(self,socketCliente):
    	self.send("Hasta luego!",socketCliente);
        socketCliente.close(); 
	os._exit(0)
    	
    
    def ejecutarComando(self,socketCliente):
	while True:
	    print "el hilo paso por el metodo"
	    comando = self.recv(socketCliente)
            if comando != '':
	        print comando
		print socketCliente
		retorno = check_output(comando)
		print retorno
		self.send(retorno,socketCliente)
    
    def run(self):
    	self.socket.listen(5)
	while True:
	    socketCliente,addr = self.socket.accept()
	    print "el servidor agarro el cliente",socketCliente
	    pid = os.fork()
            if pid == 0:
		self.ejecutarComando(socketCliente)
	    else:
        	pids = (os.getpid(), pid)
		print ("padre: %d  hijo: %d\n" % pids)

if __name__ == "__main__":
    Server().run()

    \end{python}
     \subsection{Servidor de comandos concurrente con procesos liviano(thread)}
    \begin{python}
import socket
import os
import threading
from subprocess import check_output

class Server(object):
    def __init__(self):
    	self.host, self.port = '', 9999
    	self.socket = socket.socket()
    	self.socket.bind((self.host, self.port))

    def send(self, msg,socketCliente):
    	if type(msg) == str: socketCliente.send(msg)
    	elif type(msg) == list or tuple: socketCliente.send('\n'.join(msg))

    def recv(self,socketCliente):
    	comando = socketCliente.recv(4096).strip()
        if comando == "quit":
            self.exit(socketCliente)
        else: return comando

    def exit(self,socketCliente):
    	self.send("Hasta luego!",socketCliente);
        socketCliente.close(); 
    	
    
    def ejecutarComando(self,socketCliente):
	while True:
	    print "el hilo paso por el metodo"
	    comando = self.recv(socketCliente)
            if comando != '':
	        print comando
		print socketCliente
		retorno = check_output(comando)
		print retorno
		self.send(retorno,socketCliente)
    
    def run(self):
    	self.socket.listen(5)
	while True:
	    socketCliente,addr = self.socket.accept()
	    print "el servidor agarro el cliente",socketCliente
	   
            t = threading.Thread(target=self.ejecutarComando,args=(socketCliente,))
   	    t.start()
	    print threading.currentThread()
            


if __name__ == "__main__":
    Server().run()

    \end{python}

\end{document}
