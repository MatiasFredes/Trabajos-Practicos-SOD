\documentclass[12pt,a4paper,titlepage]{article}
\usepackage[spanish]{babel}
\usepackage{listings}
\usepackage{graphicx}
\usepackage{titlepic}
\usepackage{lipsum}
\usepackage[svgnames]{xcolor}
\usepackage[tikz]{bclogo}
\usepackage{minted}

% Default fixed font does not support bold face
\DeclareFixedFont{\ttb}{T1}{txtt}{bx}{n}{12} % for bold
\DeclareFixedFont{\ttm}{T1}{txtt}{m}{n}{12}  % for normal

% Custom colors
\usepackage{color}
\definecolor{deepblue}{rgb}{0,0,0.5}
\definecolor{deepred}{rgb}{0.6,0,0}
\definecolor{deepgreen}{rgb}{0,0.5,0}

% Python style for highlighting
\newcommand\pythonstyle{\lstset{
language=Python,
basicstyle=\ttm,
otherkeywords={self},             % Add keywords here
keywordstyle=\ttb\color{deepblue},
emph={MyClass,__init__},          % Custom highlighting
emphstyle=\ttb\color{deepred},    % Custom highlighting style
stringstyle=\color{deepgreen},
frame=tb,                         % Any extra options here
showstringspaces=false            % 
}}

% Python environment
\lstnewenvironment{python}[1][]
{
\pythonstyle
\lstset{#1}
}
{}

% Python for external files
\newcommand\pythonexternal[2][]{{
\pythonstyle
\lstinputlisting[#1]{#2}}}

% Python for inline
\newcommand\pythoninline[1]{{\pythonstyle\lstinline!#1!}}


  
\title{Universidad de Mor\'on\\ Trabajos pr\'acticos de sistemas operativos distribuidos(444)}

\author{Mat\'ias Fredes \\Matricula: 38010252}
 
\begin{document}
    \maketitle
    \newpage
    \begin{bclogo}[logo=\bcattention, couleurBarre=red, noborder=true, 
               couleur=LightSalmon]{Importante!}
    Este trabajo pr\'actico fue desarrollado con el lenguaje de programaci\'on Python.\\
    Para instalar Python en su computadora debe dirigirse a la siguiente URL: https://www.python.org/downloads/.\\
    Para ejecutar los programa desde la consola, debe posicionarse en el directorio donde se encuentra el archivo y ejecutar el siguiente comando:\begin{minted}{bash}
  $ python nombreArchivo.py
    \end{minted}
    \end{bclogo}
    
    \section{Mini cliente HTTP}
    
  \noindent Aquí tenemos un Mini Cliente Apache ejecutándose en un SO Linux Debian que hace
conexi\'on con un Servidor Apache corriendo en un SO Windows XP.\\
Es muy simple, el Cliente le pide un archivo al Servidor Apache, y este le envía el
archivo al Cliente, y el Cliente lo muestra por pantalla.
El alumno tiene que modificar el Cliente para que decodifique el archivo recibido
quitando todo lo que corresponde al código html.
     
   
    \subsection{Cliente HTTP}
    \begin{python}
import socket
#gethostbyname_ex(HostName)-> Return a triple (hostname, aliaslist, ipaddrlist)
hostName = 'www.rfc-es.org'
IPhost = socket.gethostbyname_ex(hostName)[2][1]
port = 80
request = "GET / HTTP/1.1\nHost: " + hostName + "\r\n\r\n"

s = socket.socket(socket.AF_INET, socket.SOCK_STREAM)
s.connect((IPhost, port))
s.sendall(request)
print request
response = s.recv(8000)

print response
    \end{python}
  

\end{document}

